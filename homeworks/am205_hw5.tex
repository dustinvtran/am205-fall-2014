\documentclass[11pt]{article}
\usepackage{fullpage,amsmath,amsfonts,mathpazo,microtype,nicefrac,graphicx}

% Set-up for hypertext references
\usepackage{hyperref,color,textcomp}
\definecolor{webgreen}{rgb}{0,.35,0}
\definecolor{webbrown}{rgb}{.6,0,0}
\definecolor{RoyalBlue}{rgb}{0,0,0.9}
\hypersetup{
   colorlinks=true, linktocpage=true, pdfstartpage=3, pdfstartview=FitV,
   breaklinks=true, pdfpagemode=UseNone, pageanchor=true, pdfpagemode=UseOutlines,
   plainpages=false, bookmarksnumbered, bookmarksopen=true, bookmarksopenlevel=1,
   hypertexnames=true, pdfhighlight=/O,
   urlcolor=webbrown, linkcolor=RoyalBlue, citecolor=webgreen,
   pdfauthor={Chris H. Rycroft},
   pdfsubject={Harvard AM205 (Fall 2014)},
   pdfkeywords={},
   pdfcreator={pdfLaTeX},
   pdfproducer={LaTeX with hyperref}
}
\hypersetup{pdftitle={AM205: Assignment 5}}

% Macro definitions
\newcommand{\N}{\mathbb{N}}
\newcommand{\Z}{\mathbb{Z}}
\newcommand{\Q}{\mathbb{Q}}
\newcommand{\R}{\mathbb{R}}
\newcommand{\B}{\mathbb{B}}
\newcommand{\p}{\partial}
\newcommand{\Trans}{\mathsf{T}}
\renewcommand{\vec}[1]{\mathbf{#1}}
\newcommand{\vx}{\vec{x}}
\newcommand{\vb}{\vec{b}}

\DeclareMathOperator{\rank}{rank}

\begin{document}
\section*{AM205: Assignment 5 (due 5~PM, December 2nd)}
\begin{enumerate}
  \item \textbf{Rosenbrock function.} A well known benchmark problem for
    optimization algorithms is minimization of Rosenbrock's function
    \begin{equation}
      f(x,y) = 100(y-x^2)^2 + (1-x)^2,
    \end{equation}
    which has a global minimum of $0$ at $(x,y) = (1,1)$. We shall apply three
    different optimization algorithms for this function; in each case you
    should terminate the optimization algorithm when the absolute step
    size falls below $10^{-8}$.
    \begin{enumerate}
      \item Minimize Rosenbrock's function using steepest descent. You should
	try the three starting points $[-1,1]^\Trans$, $[0,1]^\Trans$, and
	$[2,1]^\Trans$, and report the number of iterations required for each
	starting point. Make a plot for each starting point that shows the
	contours of Rosenbrock's function, as well as the optimization path
	that is followed.
	
	You may use a library function for the line search in steepest descent
	if you wish. Also, note that steepest descent may require a large
	number of iterations, so you should terminate the scheme when either
	the step size tolerance (indicated above) is satisfied, or once
	2000 iterations have been performed.
      \item Repeat part (a), but with Newton's method (without line search)
	instead of steepest descent.
      \item Repeat part (a), but with BFGS instead of steepest descent. In your
	implementation of BFGS, set $B_0$ to the identity matrix.
    \end{enumerate}
  \item \textbf{Shape determination (A).} Consider a flexible
    \href{http://en.wikipedia.org/wiki/Skipping_rope}{jump rope} of length $R$
    that is initially in a vertical $xy$ plane, and hung between the points
    $(0,0)$ and $(L,0)$. Let the shape of the rope be described parametrically
    by
    \begin{equation}
      x(s) = \frac{Ls}{R} + \sum_{k=1}^8 c_k \sin \frac{\pi k s}{R}, \qquad y(s) = \sum_{k=1}^8 d_k \sin \frac{\pi k s}{R},
    \end{equation}
    where the coordinate $s$ is the distance along the string when it is
    unstretched. The jump rope is rotated around the $x$ axis with angular
    velocity $\omega$. If $\rho$ is the mass per unit unstretched length, its
    kinetic energy is
    \begin{equation}
      T= \int_0^R \rho y^2 \omega^2 ds
      \label{eq:kine}
    \end{equation}
    and its elastic potential energy is
    \begin{equation}
      V= \int_0^R \mu \left(\sqrt{\left(\tfrac{dx}{ds}\right)^2 + \left(\tfrac{dy}{ds}\right)^2}-1\right)^2 ds
      \label{eq:elae}
    \end{equation}
    for some stiffness constant $\mu$. Gravitational potential energy is
    neglected. Using the
    \href{http://en.wikipedia.org/wiki/Principle_of_least_action}{principle of
    stationary action}, the equilibrium shape of the rope will minimize $V-T$.

    Let the vector of parameters be $b=(c_1,c_2,\ldots,c_8,d_1,d_2,\ldots,d_8)$
    and let the residual be $r(b)=V-T$, where the integrals in
    Eqs.~\ref{eq:kine} and \ref{eq:elae} are evaluated at 251 control points
    using a composite integration rule of your choice.
    \begin{enumerate}
      \item Determine integral expressions for the components of $\nabla r$.
      \item Using your answer from part (a), write a program to minimize $r$
	with respect to $b$. You may use a library function from Python, Matlab,
	or other software, although you will need to write the residual
	function and its gradient. Use the parameters $R=2$, $\omega=5$, and
	$L=\rho=g=1$, and use an initial guess of $d_1=1$ with the rest of $b$
	being zero. On the same axes, plot the shape of the rope for
	$\mu=20,200,2000$.
      \item For $\mu=20,200,2000$, run your minimization algorithm starting
	from $d_2=0.5$ and all other components of $b$ being zero. Compare your
	solution with part (b).
      \item {\bf Optional.} Find two friends and a rope. Ask the two friends to
	each hold one end of the rope, and spin it between them. From a
	position perpendicular to the spinning axis, take a photo of the rope,
	trying to catch it at the moment when it is in a vertical plane.
	By choosing parameters appropriately, superpose one of your calculated
	curves from on top of the photo, and check the level of agreement. In
	addition, see if the two friends can recreate the curve from 2(c).	
    \end{enumerate}
  \item \textbf{Shape determination (B).} Now consider the case where the rope
    of length $R$ between $(0,0)$ and $(L,0)$ is inextensible. In this case,
    the rope's shape will maximize $T$ as given in Eq.~\ref{eq:kine}, subject
    to a constraint that its total length will be equal to $R$. Determine an
    appropriate parameterization of the shape of the rope, and solve the
    resulting constrained optimization problem. Find a solution that is similar
    in shape to your solution for $\mu=2000$ from question 2(b), and make a
    plot overlaying these two solutions.
  \item \textbf{Quantum eigenmodes.} Consider the one-dimensional time-independent
    Schr\"odinger equation, which governs the behavior of a quantum particle
    in a potential well. In non-dimension-alized units where \smash{$\frac{\hbar^2}{2m}=1$},
    the equation is
    \begin{equation}
      - \frac{\p^2 \Psi}{\p x^2} + v(x) \Psi(x) = E\Psi(x), \label{eq:schrod}
    \end{equation}
    where $v:\R \to \R$ is a real-valued potential fucntion, $\Psi:\R \to \R$
    is the wavefunction, and $E\in \R$ is an eigenvalue which corresponds to
    the energy of the system. In general, the wave function is complex-valued,
    but for the time-independent case it is always possible to write it as a
    real-valued function.

    The Schr\"odinger equation is posed on the infinite domain $(-\infty,\infty)$,
    and the wavefunction must satisfy $\Psi(x)\to 0$ as $x\to \pm \infty$ so
    that the norm of $\Psi$ is bounded. In this question, we shall consider the
    finite interval $[-12,12]$, which is large enough to impose zero Dirichlet
    boundary conditions at the boundaries, $\Psi(\pm 12)=0$, without
    compromising the accuracy of the results.

    As an example of a solution of Eq.~\ref{eq:schrod}, in Figure
    \ref{fig:quant} we show the first five eigenvalues and eigenmodes on $x \in
    [−12, 12$] for the Schr\"odinger solution in the case that $v(x) = x^2/10$.

    \begin{figure}
      \begin{center}
	\begin{tabular}{cc}
	  \includegraphics[width=0.5\textwidth]{quant_temp} & \parbox{3cm}{
	\begin{tabular}{|c|c|}
	  \hline
	  $E_5$ & 2.846013 \\
	  $E_4$ & 2.213572 \\
	  $E_3$ & 1.581127 \\
	  $E_2$ & 0.948679 \\
	  $E_1$ & 0.316227 \\
	  \hline
	\end{tabular}\\
	\vspace{8cm}
	\\~\\
	}
      \end{tabular}
      \end{center}
      \vspace{-6cm}
      \caption{The five lowest eigenvalues, and the corresponding eigenmodes,
      for $v(x) = x^2/10$. To show the eigenmodes in a visually appealing way
      here we have plotted $y_i(x) = 3\Psi_i(x) + E_i$ for $i = 1, \ldots ,
      5$.\label{fig:quant}}
    \end{figure}
    \begin{enumerate}
      \item Compute the five lowest eigenvalues and corresponding eigenmodes for
	the potentials
	\begin{enumerate}
	  \item $v_1(x) = |x|$,
	  \item $v_2(x) = 12 \left( \frac{x}{10} \right)^4 - \frac{x^2}{18} + \frac{x}{8} + \frac{13}{10}$,
	  \item $v_3(x) = 8|||x|-1|-1|$.
	\end{enumerate}
	You should use a second-order accurate
	finite-difference apporoximation of Schr\"odinger equation with
	$n=1921$ grids points on the interval $[-12,12]$, end then employ an
	eigensolve such as the Python/Matlab \texttt{eig}/\texttt{eigs}
	routines. Impose zero boundary conditions at $x=\pm 12$ as described
	above. Present your results using a figure and a table in the same way
	as in Figure \ref{fig:quant}. 
      \item Quantum mechanics tells us that if a particle has a wavefunction $\Psi(x)$,
	the the probability of finding it in a region $[a,b]$ is given by
	\begin{equation}
	  \frac{\int_a^b |\Psi(x)|^2 dx}{\int_{-\infty}^\infty |\Psi(x)|^2 dx}.
	\end{equation}
	For $[a,b] \subset [-12,12]$ this can be approximated on the finite
	grid as
	\begin{equation}
	  \frac{\int_a^b |\Psi(x)|^2 dx}{\int_{-12}^{12} |\Psi(x)|^2 dx}.\label{eq:qprob}
	\end{equation}
	For each of the first five eigenmodes for the potential $v_2$, use the
	composite Simpson rule and Eq.~\ref{eq:qprob} to compute the probability that the
	particle is in the region $x \in [0,6]$ (\textit{i.e.} specify five different
	probabilities, one corresponding to each eigenmode). When you use the
	composite Simpson rule here, you should use all grid points from (a)
	that are inside the interval of interest as quadrature points.
      \item \textbf{Optional.} Modify your program from part (a) to use
	fourth-order accurate finite differences, using the stencils described
	\href{http://en.wikipedia.org/wiki/Finite_difference_coefficient}{on
	the web}, with suitable modifications at the end points.
    \end{enumerate}
\end{enumerate}
\end{document}
