\documentclass[11pt]{article}
\usepackage{fullpage,amsmath,amsfonts,mathpazo,microtype,nicefrac,graphicx}

% Set-up for hypertext references
\usepackage{hyperref,color,textcomp}
\definecolor{webgreen}{rgb}{0,.35,0}
\definecolor{webbrown}{rgb}{.6,0,0}
\definecolor{RoyalBlue}{rgb}{0,0,0.9}
\hypersetup{
   colorlinks=true, linktocpage=true, pdfstartpage=3, pdfstartview=FitV,
   breaklinks=true, pdfpagemode=UseNone, pageanchor=true, pdfpagemode=UseOutlines,
   plainpages=false, bookmarksnumbered, bookmarksopen=true, bookmarksopenlevel=1,
   hypertexnames=true, pdfhighlight=/O,
   urlcolor=webbrown, linkcolor=RoyalBlue, citecolor=webgreen,
   pdfauthor={Chris H. Rycroft},
   pdfsubject={Harvard AM205 (Fall 2014)},
   pdfkeywords={},
   pdfcreator={pdfLaTeX},
   pdfproducer={LaTeX with hyperref}
}
\hypersetup{pdftitle={AM205: Assignment 3}}

% Macro definitions
\newcommand{\N}{\mathbb{N}}
\newcommand{\Z}{\mathbb{Z}}
\newcommand{\Q}{\mathbb{Q}}
\newcommand{\R}{\mathbb{R}}
\newcommand{\C}{\mathbb{C}}
\newcommand{\B}{\mathbb{B}}
\newcommand{\p}{\partial}
\newcommand{\Trans}{\mathsf{T}}
\renewcommand{\vec}[1]{\mathbf{#1}}
\newcommand{\vx}{\vec{x}}
\newcommand{\vb}{\vec{b}}

\DeclareMathOperator{\rank}{rank}

\begin{document}
\section*{AM205: Assignment 3 (due 5~PM, October 24)}
\begin{enumerate}
  \item {\bf Convergence rates of two integrals.} Consider the function
    \begin{equation}
      f(x)=\frac{1}{\tfrac{5}{4} - \cos x}.
    \end{equation}
    \begin{enumerate}
      \item It can be shown that
	\begin{equation}
	  I_A = \int_0^{\pi/3}f(x)dx = \frac{8\pi}{9}.
	\end{equation}
	Write a program to numerically evaluate $I_A$ using the composite
	trapezoid rule with $n$ intervals of equal size $h$, for
	$n=1,2,\ldots,50$, and make a log--log plot of the absolute error as a
	function $h$. On the same axes, overlay the composite trapezoid error
	bound from the lectures,
	\smash{$E(h)=\frac{h^2\pi}{24}||f''||_\infty$}, and show that your
	numerically computed results are smaller than the bound.
      \item It can be shown that
	\begin{equation}
	  I_B = \int_0^{2\pi}f(x)dx = \frac{8\pi}{3}.\label{eq:res_calc}
	\end{equation}
	Write a program to numerically evaluate $I_B$ using the composite
	trapezoid rule with $n$ points, for $n=1,2,\ldots,50$. Make a log--log
	plot of the absolute error as a function $h$. Does the absolute error
	scale like $h^m$ for some $m$?
      \item \textbf{Optional.} Use
	\href{http://en.wikipedia.org/wiki/Residue_theorem}{residue calculus}
	to prove the integral in Eq.~\ref{eq:res_calc}.
    \end{enumerate}
  \item
    \begin{enumerate}
      \item \textbf{Adaptive integration.} Given that the cubic Legendre
	polynomial is \smash{$P_3(x) = \frac{1}{2}x(5x^2-3)$}, derive the
	3-point Gauss quadrature rule on the interval $[-1,1]$ by evaluating
	the relevant integrals by hand. Demonstrate that this quadrature rule
	integrates all polynomials up to the expected degree exactly.
      \item In the lectures we discussed a method of evaluating integrals
	\smash{$\int_a^b f(x)dx$} by adaptively refining the calculation in
	regions where the function varies rapidly. To begin, a tolerance level
	$T$ is introduced, and the calculation starts using a single
	integration interval $[a,b]$.
	
	Let $I_{a,b}$ be the integral of $f$ over this interval using the
	three-point quadrature rule from part (a). In addition, define
	\smash{$c=\frac{a+b}{2}$} and calculate $\hat{I}_{a,b}=I_{a,c}+I_{c,b}$
	as a more refined estimate of the integral. Then
	$E_{a,b}=|I_{a,b}-\hat{I}_{a,b}|$ is an estimate of the error of
	$I_{a,b}$. If $E_{a,b}<Tl$, where $l$ is the length of the interval,
	then the error is acceptable, and the method can terminate. Otherwise
	this interval must be subdivided into $[a,c]$ and $[c,b]$, and the
	above procedure must be applied to these two intervals. The procedure
	must be applied recursively until the errors become smaller than the
	tolerance.

	Write a function to implement this adaptive integration scheme. Using
	$T=10^{-6}$, apply it to the integrals
	\begin{equation}
	  \int_{-1}^{3/4} (x^n-x^2+1) dx
	\end{equation}
	for $m=4,5,6,7,8$. For each case, report the value of the integral, the
	total estimated error (by summing the relevant $E_{\alpha,\beta}$
	terms), and the total number of intervals that are used.
      \item Use your adaptive integration routine from part (a) and $T=10^{-6}$
	to evaluate the four integrals
	\[
	\int_{-1}^1 |x| dx, \qquad \int_{-1}^2 |x| dx, \qquad \int_{-1}^2 \cosh\left(\sin \frac{x\pi \sqrt{5}}{\sqrt{3}}\right) dx, \qquad \int_0^1 x^{3/4} \sin \frac{1}{x} \, dx.
	\]
	For each case, report the value of the integral, the
	total estimated error (by summing the relevant $E_{\alpha,\beta}$
	terms), and the total number of intervals that are used.
      \item One of your integrals from part (c) is incorrect, and the true
	answer is significantly outside the estimated error bound. Determine
	which integral this is, and state why this happens. Find a method of
	altering your integration routine to correctly calculate the integral
	to five decimal places of accuracy.
    \end{enumerate}
  \item \textbf{Error analysis of a numerical integration rule.} Applying the
    midpoint quadrature rule (\textit{i.e.} $n=0$ Newton--Cotes with the quadrature
    point at the midpoint) on the interval $[t_k,t_{k+1}]$ to \smash{$y(t_{k+1}) =
    y(t_k) + \int_{t_k}^{t_{k+1}}f(t,y(t))dt$} leads to the implicit
    \textit{midpoint method},
    \begin{equation}
      y_{k+1} = y_{k} + hf(t_{k+1/2},(y_k+y_{k+1})/2),
    \end{equation}
    where \smash{$t_{k+1/2} = t_k + \frac{h}{2}$}.
    \begin{enumerate}
      \item Use Taylor series expansions to show that the order of accuracy of
	this method is 2.
      \item What is the stability region of the method for the equation $y' =
	\lambda y$? In other words, for what values of $\bar{h} = h\lambda \in
	\C$ is the method stable?
    \end{enumerate}
  \item 
    \begin{enumerate}
      \item \textbf{A multi-step method.} Consider solving the differential
	equation $y'=f(t,y)$ at timepoints $t_k$ with corresponding numerical
	solutions $y_k$. The multi-step Nystr\"om numerical method is based
	upon the integral relation
	\begin{equation}
	  y(t_{k+1}) = y(t_{k-1}) + \int_{t_{k-1}}^{t_{k+1}} f(t,y) dt.
	\end{equation}
	Derive a multi-step numerical method by approximating the integrand
	$f(t,y)$ with the polynomial interpolation using the function
	values at $t_{k-2}$, $t_{k-1}$, and $t_k$. Your method should have the
	form
	\begin{equation}
	  y_{k+1} = y_{k-1} + h(\alpha f_{k-2}+\beta f_{k-1} + \gamma f_k) \label{eq:nys}
	\end{equation}
	where $\alpha$, $\beta$, and $\gamma$ are constants to be determined,
	$h$ is the timestep interval size, and $f_l=f(t_l,y_l)$ for an arbitrary
	$l$.
      \item Find the exact solution to the second-order differential equation
	\begin{equation}
	  y''(t)+2y'(t) + 5y(t)=0. \label{eq:ode2}
	\end{equation}
	subject to the initial conditions $y(0)=1$, $y'(0)=0$.
      \item Write Eq.~\ref{eq:ode2} as a coupled system of two first-order
	differential equations for $\vec{y}=(y,v)=(y,y')$. Solve the system
	over the interval $0\le t\le 3$ with a timestep of $h=0.02$ using your
	multi-step method from part (a).

	Before Eq.~\ref{eq:nys} can be applied, $\vec{y}_1$ and $\vec{y}_2$
	must be calculated accurately. Use one of the following two approaches:
	\begin{enumerate}
	  \item set them based on the exact solution from part (b),
	  \item calculate them using the classical fourth-order Runge--Kutta
	    method.
	\end{enumerate}
	Make a log--log plot of the absolute error between the numerical and
	exact values of $y$ at $t=3$ as a function of $h$, over the range from
	$h=10^{-4}$ to $h=10^{-1}$. Show that your method is third-order
	accurate.
      \item \textbf{Optional.} Suppose that instead of setting $\vec{y}_1$ and
	$\vec{y}_2$ accurately, you instead make use of forward Euler steps.
	Create a log--log plot of the absolute error as a function of $h$ for
	this case and determine the order of accuracy.
    \end{enumerate}
  \item \textbf{Asteroid collision.} An asteroid is detected near the Earth and
    the Moon in the plane of their orbits. Use your ODE-solving skills to
    determine if there is danger of collision.

    This problem can be treated as a circular restricted three-body problem,
    which considers the motion of an object of negligible mass in the presence
    of two massive gravitating bodies that circularly orbit each other. The
    motion of the object is assumed to be restriced to the 2D plane of the
    circular orbit of the two massive bodies.

    The Earth and Moon have circular orbits about each other. If we work in a
    co-rotating frame, we can consider the Earth and Moon to be at fixed
    locations and then we only need to determine the trajectory of the
    asteroid. In dimensionless units, fix the Earth at position $(x,y)=(0,0)$
    and the Moon at position $(x,y)=(1,0)$. 

    The asteroid (which has negligible mass) will move according to the
    gravitational forces it feels from the Earth and the Moon. The motion of
    the asteroid is described by its position $x(t)$, $y(t)$, and velocity
    $u(t)$, $v(t)$ as functions of time. Throughout the asteroid's movement,
    the \textit{Jacobi integral}, $J$, is a constant of motion, and is given
    by
    \begin{equation}
      J(x,y,u,v) = (x+\mu)^2 + y^2 + \frac{2(1-\mu)}{\sqrt{x^2+y^2}} +
      \frac{2\mu}{\sqrt{(x-1)^2+y^2}} - u^2 - v^2
    \end{equation}
    where $\mu=0.01$ is the mass ratio of the Moon to Earth.
    The asteroid's equations of motion are given by the system of ODEs,
    \begin{align}
      x' &= -\frac{1}{2}\frac{\p J}{\p u}, &y' &= -\frac{1}{2}\frac{\p J}{\p v}, \nonumber\\
      u' &= v +\frac{1}{2}\frac{\p J}{\p x},  &v' &= -u+\frac{1}{2}\frac{\p J}{\p y}.
    \end{align}
    \begin{enumerate}
      \item Write the system of ODEs to be solved in terms of $x$, $y$, $u$,
	and $v$.
      \item Suppose that based on observations we know that at $t=0$ the
	asteroid's position and velocity are given by $(x(0),y(0))=(1.08,0)$
	and $(u(0),v(0)) = (0.5\cos(\theta),0.5\sin(\theta))$ where
	\smash{$\theta\in[\frac{3\pi}{2}-0.3,\frac{3\pi}{2}+0.3]$}, so there is some
	uncertainty in the asteroid's initial velocity. Use your favorite
	ODE solver\footnote{In Python, a good choice would be the \texttt{odeint} routine.
	In MATLAB, the \texttt{ode45} routine could be used, using the additional
	command \texttt{odeset('RelTol',1e-10,'AbsTol',1e-10)} to improve accuracy.}
	to integrate this ODE from $t=0$ to $t=7$, using 5 different
	initial conditions determined by sampling $\theta$ at evenly spaced
	points in the interval
	\smash{$[\frac{3\pi}{2}-0.3,\frac{3\pi}{2}+0.3]$} (your samples should
	include the endpoints of the interval).

	On a single figure, plot the five trajectories $(x(t),y(t))$ of the
	asteroid corresponding to the initial conditions you considered. The
	Earth has radius $R_\text{Earth}=0.02$ and the Moon has radius
	$R_\text{Moon}=0.005$. Add circles of the appropriate radii on your
	plot to indicate the regions occupied by the Moon and the Earth. Do any
	of the five trajectories collide with the Earth or Moon between $t=0$
	and $t=7$? If yes, which initial condition (or initial conditions)
	produce a collision?
      \item If this ODE system were solved exactly, the Jacobi integral would
	be exactly constant as a function of time, since it is a conserved
	quantity of the system. We can therefore use the value of the Jacobi
	integral to check the robustness of our numerical approximation. Plot
	the relative error in the Jacobi integral, \smash{$\frac{|J(t)-J(0)|}{J(0)}$},
	for $t \in [0,7]$ for the \smash{$\theta=\frac{3\pi}{2}$} trajectory and
	explain what you observe.
      \item \textbf{Optional.} Try some alternative ODE solution methods and
	compare the errors in $J$ that occur over the integration interval.
    \end{enumerate}
\end{enumerate}
\end{document}
