\documentclass[12pt]{article}
\usepackage{fullpage,amsmath,amsfonts,mathpazo,microtype,nicefrac}

% Set-up for hypertext references
\usepackage{hyperref,color,textcomp}
\definecolor{webgreen}{rgb}{0,.35,0}
\definecolor{webbrown}{rgb}{.6,0,0}
\definecolor{RoyalBlue}{rgb}{0,0,0.9}
\hypersetup{
   colorlinks=true, linktocpage=true, pdfstartpage=3, pdfstartview=FitV,
   breaklinks=true, pdfpagemode=UseNone, pageanchor=true, pdfpagemode=UseOutlines,
   plainpages=false, bookmarksnumbered, bookmarksopen=true, bookmarksopenlevel=1,
   hypertexnames=true, pdfhighlight=/O,
   urlcolor=webbrown, linkcolor=RoyalBlue, citecolor=webgreen,
   pdfauthor={Chris H. Rycroft},
   pdfsubject={Harvard AM205 (Fall 2014)},
   pdfkeywords={},
   pdfcreator={pdfLaTeX},
   pdfproducer={LaTeX with hyperref}
}
\hypersetup{pdftitle={AM205: Assignment 1}}

% Macro definitions
\newcommand{\N}{\mathbb{N}}
\newcommand{\Z}{\mathbb{Z}}
\newcommand{\Q}{\mathbb{Q}}
\newcommand{\R}{\mathbb{R}}
\newcommand{\p}{\partial}
\newcommand{\Trans}{\mathsf{T}}

\begin{document}
\section*{AM205: Assignment 1 (due 5~PM, September 26)}
\begin{enumerate}
  \item
    \begin{enumerate}
      \item Derive the interpolating cubic polynomial for the data points
	$(0,0)$, $(1,0)$, $(2,1)$, and $(3,2)$ using the monomial basis and the
	Lagrange basis. Show that these two representations are equivalent.
	{\it You may need to use NumPy, MATLAB, or other software to solve the
	linear system for the four degrees of freedom.}
      \item Find the interpolating polynomial for the data points $(0,2)$, $(2,0)$,
	$(4,0)$, and $(6,0)$ either by direct calculation or by using your result
	from (a).
    \end{enumerate}
  \item 
    \begin{enumerate}
      \item {\bf Error bounds with Lagrange polynomials.} Let
	$f(x)=e^{4x}+e^{-2x}$. Write a program to calculate and plot the
	Lagrange polynomial $p_{n-1}(x)$ of $f(x)$ at the Chebyshev points
	$x_j = \cos ((2j-1)\pi/2n)$ for $j = 1,...,n$. For $n=4$, over the
	range $[-1,1]$, plot $f(x)$ and Lagrange polynomial $p_3(x)$.
      \item Recall from the lectures that the infinity norm for a function $g$
	on $[-1,1]$ is defined as \smash{$||g||_\infty = \max_{x\in[-1,1]}
	|g(x)|$}. Calculate $||f-p_3||_\infty$ by sampling the function at
	1,000 equally-spaced points over $[-1,1]$.
      \item Recall the interpolation error formula from the lectures,
	\begin{equation}
	  f(x)-p_{n-1}(x) = \frac{f^{(n)}(\theta)}{n!} (x-x_1)(x-x_2)\ldots(x-x_n)
        \end{equation}
	for some $\theta \in [-1,1]$. Use this formula to derive an upper
	bound for $||f-p_{n-1}||_\infty$ for any positive integer $n$. Your
	bound should be a mathematical formula, and should not rely on
	numerical sampling.
      \item Find a cubic polynomal $p_3^\dagger$ such that
	$||f-p_3^\dagger||_\infty<||f-p_3||_\infty$.
    \end{enumerate}
  \item {\bf Periodic cubic splines.} In the lectures we discussed the
    construction of cubic splines to interpolate between a number of control
    points. We found that it was necessary to impose additional constraints at
    the end points of the spline in order to have enough constraints to
    determine the cubic spline uniquely. Here, we examine the construction of
    cubic splines on a periodic interval $t\in [0,4)$, where $t=0$ is
    equivalent to $t=4$. By working in a periodic interval, this
    simplifies the spline construction and no additional end point constraints
    are required.
    \begin{enumerate}
      \item Consider the four points $(x,t)=(0,0), (1,1), (2,0), (3,-1)$.
	Construct a cubic spline $s_x(t)$ that is piecewise cubic in the four
	intervals $[0,1)$, $[1,2)$, $[2,3)$, and $[3,4)$. At $t=0,1,2,3$ the
	cubics should match the control points, giving eight constraints. At
	$t=0,1,2,3$ the first and second derivatives should match, giving and
	additional eight constraints and allowing $s_x(t)$ to be uniquely
	determined.
      \item Plot $s_x(t)$ and $\sin(t\pi/2)$ on the interval $[0,4)$ and
	show that they are similar.
      \item Construct a second cubic spline $s_y(t)$ that goes through the
	four points $(0,1)$, $(1,0)$, $(2,-1)$, and $(3,0)$. Plot $s_y(t)$ and $\cos(t\pi/2)$
	on the interval $0 \le t <4$ and show that they are similar.
      \item In the $xy$-plane, plot the parametric curve $(s_x(t),s_y(t))$ for $t\in[0,4)$.
	Calculate the area enclosed by the parametric curve, and use it to
	estimate $\pi$ to at least five decimal places, using the relationship
	$A=\pi r^2$ where $r$ is taken to be 1.
      \item {\bf Optional for the enthusiasts.} In parts (a) to (c), you
	constructed cubic spline approximations for sine and cosine using four
	equally-spaced control points. Generalize this to construct cubic spline
	approximations for sine and cosine using $n$ equally-spaced control
	points. What is the rate of convergence of the approximation of $\pi$
	as $n$ is increased?
    \end{enumerate}
  \item {\bf Fitting a planet's orbit.} A planet follows an elliptical orbit,
    which can be represented in a Cartesian $(x,y)$ coordinate system by the
    equation
    \begin{equation}
      b_0 + b_1x + b_2 y + b_3 xy + b_4 y^2 = x^2.
    \end{equation}
    \begin{enumerate}
      \item Use a linear least-squares fit for the parameters $b_0, b_1, b_2,
	b_3, b_4$, given the following observations of the planet's position:
	\begin{center}
	  \begin{tabular}{c|ccccc}
	    $x$ & 1.02 & 0.95 & 0.87 & 0.77 & 0.67 \\
	    $y$ & 0.39 & 0.32 & 0.27 & 0.22 & 0.18 \\
	    \hline
	    $x$ & 0.56 & 0.44 & 0.30 & 0.16 & 0.01 \\
	    $y$ & 0.15 & 0.13 & 0.12 & 0.13 & 0.15
	  \end{tabular}
	\end{center}
	This data is provided in the file \texttt{q4a\_data.txt} on the AM205
	website.) Plot the resulting orbit with a continuous curve (making sure
	that you plot the entire ellipse) and the observations (marked with
	$\times$ symbols) on the same plot. What is the 2-norm of the residual
	for your best fit?
      \item The observation data is nearly rank-deficient, which implies that
	the matrix $A^\Trans A$ is nearly singular and hence the parameter fit will
	be sensitive to perturbations in the data ({\it i.e.} the least-squares
	fit is poorly conditioned in this case). To show this, compute the best
	fit to the perturbed data $\hat x = x + \Delta x$ and $\hat y = y +
	\Delta y$, where $\Delta x$, $\Delta y$ are given below.
	\begin{center}
	  \begin{tabular}{c|ccccc}
	    $\Delta x$ & -0.0029 & 0.0007 & -0.0082 & -0.0038 & -0.0041\\
	    $\Delta y$ & -0.0033 & 0.0043 & 0.0006 & 0.0020 & 0.0044 \\
	    \hline
	    $\Delta x$ &  0.0026 & -0.0001 & -0.0058 & -0.0005 & -0.0034 \\
	    $\Delta y$ & 0.0009 &  0.0028 & 0.0034 & 0.0059 & 0.0024
	  \end{tabular}
	\end{center}
	This data is provided in the file \texttt{q4b\_data.txt} on the AM205 website.
	Overlay the two sets of observations and corresponding orbits on the
	same plot.
    \end{enumerate}
  \item {\bf Approximate solution of a partial differential equation using
    least-squares fitting.} In the $xy$-plane, an elastic string is stretched
    along the horizontal axis between $x=0$ and $x=\pi$. It undergoes small
    vibrations in the vertical direction, so that its vertical position is
    given by the function $y(x,t)$. The function satisfies the wave equation
    \begin{equation}
      \frac{\p^2 y}{\p t^2} = c^2 \frac{\p^2 f}{\p x^2},
      \label{eq:wave}
    \end{equation}
    for some parameter $c$, subject to the boundary conditions that
    $y(0,t)=y(\pi,t)=0$. One can verify that
    \begin{equation}
      y_k(x,t)= \sin kx \cos kct
    \end{equation}
    is a solution to Eq.~\ref{eq:wave} for any integer $k$. Since
    Eq.~\ref{eq:wave} is linear in $y$, it follows that any linear combination
    of the $y_k$ ({\it e.g.} $y_1+0.2y_2-0.3y_{6}$) will also be a solution.
    \begin{enumerate}
      \item Suppose that the initial condition is
	\begin{equation}
	  y(x,0) = 10^{-4}x^{8}(\pi-x)^4
	  \label{eq:init}
	\end{equation}
	and $\p_t y(x,0)=0$. Consider the approximation
	\begin{equation}
	  y(x,0) \approx \sum_{k=1}^6 \alpha_k \sin kx
	  \label{eq:approx}
	\end{equation}
	for some parameters $\alpha_1, \alpha_2, \ldots, \alpha_6$. By
	evaluating Eq.~\ref{eq:init} at the 49 points \smash{$\frac{\pi}{50}$},
	\smash{$\frac{2\pi}{50}$}, $\ldots$, \smash{$\frac{49\pi}{50}$}, find
	the least-squares fit of the $\alpha_i$.
      \item Using your calculation from part (a), write down an approximate
	solution $y_\text{ap}(x,t)$ to Eq.~\ref{eq:wave}. Set $c=1$. On the
	same axes, plot Eq.~\ref{eq:init}, and $y_\text{ap}(x,t)$ for
	\smash{$t=0,\frac{\pi}{4}, \frac{\pi}{2}, \frac{3\pi}{4}, \pi$}.
      \item {\bf Optional for the enthusiasts.} If you are familiar with
	Fourier analysis, you will notice that Eq.~\ref{eq:approx} is similar
	to a Fourier sine series,
	\begin{equation}
	  y(x,0) = \sum_{k=1}^\infty b_k \sin kx.
	\end{equation}
	Evaluate the $b_k$ using the standard Fourier integral formula---this
	can either be done numerically or with the aid of a symbolic
	maninupulation package such as Maple or Mathematica. Compare the values
	of the $b_k$ with the $\alpha_k$.
    \end{enumerate}
\end{enumerate}
\end{document}
